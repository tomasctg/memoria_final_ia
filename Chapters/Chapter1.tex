% Chapter 1

\chapter{Introducción} % Main chapter title

\label{Chapter1} % For referencing the chapter elsewhere, use \ref{Chapter1} 
\label{IntroGeneral}

En este capítulo se presenta una introducción general al proyecto. Se describe la problemática fundamental del riego solar que motivó este trabajo y se analiza el estado del arte en las áreas clave del modelado de sistemas y su optimización. Finalmente, se expone la motivación de la solución propuesta y se delimitan los objetivos y el alcance del software que fue desarrollado.

%----------------------------------------------------------------------------------------

% Define some commands to keep the formatting separated from the content 
\newcommand{\keyword}[1]{\textbf{#1}}
\newcommand{\tabhead}[1]{\textbf{#1}}
\newcommand{\code}[1]{\texttt{#1}}
\newcommand{\file}[1]{\texttt{\bfseries#1}}
\newcommand{\option}[1]{\texttt{\itshape#1}}
\newcommand{\grados}{$^{\circ}$}

%----------------------------------------------------------------------------------------

%\section{Introducción}

%----------------------------------------------------------------------------------------
\section{Contexto y problemática del riego solar}

La implementación de sistemas de riego 100\% solares es una solución tecnológica de amplio uso en el sector agrícola, particularmente en plantaciones que operan de forma autónoma (es decir, en modo \textit{off-grid}), donde la red eléctrica de distribución pública no está disponible o su conexión resulta económicamente inviable. Estas instalaciones se componen típicamente de una \textbf{Capa Física} (la bomba sumergible y las válvulas de campo) y una \textbf{Capa de Control y Adquisición} (el variador de frecuencia (VFD) y el PLC que comandan dicho hardware). En la operación estándar, esta arquitectura de control es robusta para ejecutar comandos, pero carece de la capa superior de optimización que define el plan de operación. 

Esta ausencia de una capa de gestión inteligente traslada por completo la carga de optimización al operario. El despacho de agua se convierte en una consecuencia pasiva de la meteorología —siguiendo la curva de irradiación solar— o, en el mejor de los casos, sigue un plan estático (basado en temporizadores) definido manualmente por el ingeniero agrónomo. Este enfoque manual representa una carga significativa de horas de ingeniería y es inherentemente subóptimo, ya que no puede reaccionar a la variabilidad diaria de las condiciones hídricas o energéticas. 

La necesidad de esta optimización inteligente se vuelve crítica en contextos agrícolas como el del cultivo de pistacho, donde los ciclos de inversión son largos y la planta requiere varios años para entrar en producción. En regiones donde el agua es un recurso escaso o sujetas a sequías estacionales, una gestión deficiente del riego no solo reduce la eficiencia, sino que compromete la viabilidad y sostenibilidad a largo plazo de la plantación. 

Por lo tanto, la problemática central que este trabajo aborda es esta falta de inteligencia operativa. La proliferación de la telemetría y los sensores de campo (IoT) en la agricultura moderna han generado un vasto potencial de optimización que, en la mayoría de los sistemas de riego, permanece sin explotar \cite{IoTReviewMDPI}. Los controladores actuales se limitan a registrar estos datos o a reaccionar a umbrales simples, pero carecen de la capacidad de utilizarlos para una planificación proactiva.

El verdadero desafío que se resuelve en este trabajo es integrar la Inteligencia Artificial para dos tareas claves: primero, para transformar este flujo de datos en conocimiento predictivo (un modelo de alta fidelidad del sistema); y segundo, para actuar sobre dicho conocimiento. Se requiere un motor de optimización —en este caso, un Algoritmo Genético— que funcione como un buscador capaz de \textit{generar} un cronograma de despacho óptimo a largo plazo. Esta capacidad de búsqueda y planificación proactiva, que balancea múltiples objetivos, es la pieza de gestión que falta en los sistemas actuales, los cuales obligan a operar muy por debajo de su potencial de eficiencia. 

%----------------------------------------------------------------------------------------

\section{Estado del arte: modelado y optimización}

El problema central de este trabajo —generar un cronograma de despacho de riego óptimo para un sistema solar autónomo— no es un desafío singular, sino la intersección de dos conceptos técnicos que el estado del arte ha abordado históricamente por vías separadas:
\begin{enumerate}
    \item \textbf{El desafío del modelado:} se requiere un simulador de muy alta fidelidad, un ``gemelo digital``, que actúe como la función de evaluación del optimizador. Este modelo debe predecir con precisión la respuesta hidráulica (caudales, presiones, nivel freático) ante cualquier plan de riego candidato. 
    \item \textbf{El desafío de la optimización:} se necesita un algoritmo de búsqueda capaz de encontrar la combinación óptima de decisiones (qué válvula abrir y cuándo) dentro de un espacio de soluciones de dimensionalidad masiva, respetando un conjunto de restricciones físicas (energía, agua, presión) y un tiempo de cómputo estricto. 
\end{enumerate}

A continuación, se analiza el panorama de las soluciones existentes para cada uno de estos desafíos, identificando las limitaciones que motivan el desarrollo de este trabajo. 

\subsection{El modelado: consistencia vs. precisión }

El algoritmo de optimización es, por naturaleza, agnóstico a la física del problema; depende estrictamente de una \textbf{función de evaluación} (en este caso, el gemelo digital) para cuantificar la calidad, o \textit{fitness}, de un cronograma de riego candidato. El éxito de la optimización, por lo tanto, depende enteramente de la fidelidad de este modelo. El enfoque tradicional para construir esta función se basa en modelos de \textbf{``caja blanca``} o de primeros principios, como el estándar industrial EPANET, que resuelven las ecuaciones fundamentales de la mecánica de fluidos \cite{DITEC-WNTR, WNTR-Python}. Su fortaleza indiscutible es la \textbf{consistencia física}: al estar basados en leyes universales, son robustos y pueden generalizar su comportamiento a escenarios de operación novedosos \cite{HybridPatternsArxiv}. Sin embargo, su debilidad es la \textbf{incertidumbre }en sus parámetros físicos; el rendimiento del modelo depende de una calibración precisa de variables estáticas (ej. rugosidad de tuberías) que son difíciles de medir y cambian con el tiempo \cite{CalibrationModelANN}. Esto genera un \textbf{error sistemático o ``residual``} entre la simulación idealizada y la realidad del campo. 

Para solucionar esta brecha de precisión, en el extremo opuesto del espectro se encuentran los modelos puramente empíricos de \textbf{``caja negra``} (IA), que buscan aprender el comportamiento del sistema basándose únicamente en datos históricos de telemetría. La ventaja de este enfoque es la capacidad de alcanzar una alta precisión, capturando dinámicas complejas que el modelo físico ignora \cite{HybridReviewTandF}. No obstante, esta solución es \textbf{frágil}: requiere volúmenes masivos de datos para cubrir todo el espacio operativo \cite{ReviewPhysicsInformed}, no es robusta para extrapolar a escenarios no vistos y, crucialmente, no garantiza la consistencia física \cite{ReviewPhysicsInformed}.  

El panorama de las soluciones de modelado presenta, por lo tanto, un dilema fundamental: el ingeniero debe elegir entre la \textbf{consistencia física} de la ``caja blanca`` (sufriendo su imprecisión) o la \textbf{precisión} de la ``caja negra`` (sufriendo su falta de robustez). Esta dicotomía, junto con la alternativa conceptual de un modelo híbrido o de ``caja gris`` que busca unificar ambas \citep{GreyBoxWikipedia, GreyBoxHydrology}, se ilustra en la figura \ref{fig:diagrama_caja_negra_blanca}. 

\begin{figure}[ht]
	\centering
	\includegraphics[width=\textwidth]{Figures/diagrama_cajas-2.pdf}
	\caption{Diagrama conceptual de los paradigmas de modelado: caja blanca (física), caja negra (IA pura) y caja gris (híbrido).}
	\label{fig:diagrama_caja_negra_blanca}
\end{figure}

\subsection{Optimización: optimalidad vs. factibilidad }

El segundo desafío fundamental radica en el propio algoritmo de búsqueda. La elección de la herramienta de optimización no es arbitraria, sino que está dictada por la naturaleza matemática inherente al problema del despacho de riego. Al formularlo, el problema se revela como un \textbf{Problema de Programación Mixta-Entera No Lineal (MINLP)}, una clasificación con profundas implicaciones computacionales.

 La naturaleza \textbf{Mixta-Entera (MI)} surge de las variables de decisión: el estado de una válvula o bomba en un intervalo de tiempo 
  es una decisión fundamentalmente binaria (0 para cerrado o 1 para abierto). A esto se suma la \textbf{No Linealidad (NL)}, que proviene directamente de la función de evaluación. El modelo hidráulico que predice el caudal y la presión —ya sea de caja blanca o negra— es una función compleja de la física de fluidos, no una simple suma algebraica. 

 La combinación de decisiones binarias (un espacio de búsqueda combinatorio) y una función de evaluación no lineal clasifica este problema en la categoría de complejidad \textbf{NP-hard} \citep{GAschedulingSussex}. Esto significa que es computacionalmente intratable: el tiempo requerido para encontrar la solución óptima garantizada crece exponencialmente a medida que el problema escala (más válvulas o más intervalos). El espacio de soluciones a explorar es demasiado grande.

Esta realidad computacional genera un conflicto directo e irresoluble con las necesidades de un sistema de riego operativo. El estado del arte, si bien ofrece \textit{solvers} exactos para MINLP que buscan el óptimo global, no puede garantizar el tiempo de convergencia. 

Dado este compromiso entre la optimalidad y la factibilidad, la práctica establecida para problemas de \textit{scheduling} (programación) y asignación de recursos NP-hard es el uso de \textbf{metaheurísticas}\citep{GAschedulingSurvey, PerformanceMetaheuristics}. Algoritmos como los Algoritmos Genéticos (GA) proponen un enfoque alternativo: en lugar de buscar una garantía de optimalidad global, ofrecen una \textbf{garantía de tiempo de ejecución}. Están diseñados para encontrar soluciones de alta calidad (casi óptimas) dentro de un presupuesto computacional fijo \citep{ComparativeMetaheuristics}. Para un problema de ingeniería en un entorno productivo, esta capacidad de obtener una solución factible en un tiempo acotado es un requisito operativo fundamental, convirtiendo a esta estrategia en un enfoque viable.
%----------------------------------------------------------------------------------------

\section{Motivación}

La motivación de este trabajo final es proponer una arquitectura de software unificada que resuelva los dos dilemas fundamentales identificados en el estado del arte. Este proyecto se fundamenta en la hipótesis de que es posible lograr tanto la precisión en el modelado como la factibilidad en la optimización, sin sacrificar la robustez física ni el rendimiento operativo.
Ante el desafío del modelado, que obliga a elegir entre la consistencia de la ``caja blanca`` y la precisión de la ``caja negra``, este trabajo implementa un enfoque de ``caja gris`` o Gemelo Digital Híbrido. Esta arquitectura utiliza el simulador físico (basado en EPANET/WNTR) para obtener una línea de base robusta y físicamente coherente. Adicionalmente, emplea un modelo de Inteligencia Artificial (IA) no para modelar el sistema completo, sino con la tarea específica de predecir y corregir el error residual del modelo físico \citep{HybridModelsKaggle, HybridSeriesArxiv}. De esta forma, se busca unificar la consistencia física de los primeros principios con la precisión empírica de los modelos de IA.

El desafío de la optimización (MINLP) también requiere un enfoque específico. Si bien los solvers exactos son la herramienta fundamental para garantizar la optimalidad global, su tiempo de convergencia es, por la naturaleza NP-hard del problema, indefinido. Un cálculo que puede tardar horas o días no es una herramienta útil para la toma de decisiones agronómicas diarias.
Por esta razón, el proyecto encontró un enfoque alternativo que prioriza la velocidad de respuesta. Se seleccionó una metaheurística, el Algoritmo Genético (GA), fundamentada en su probada robustez para problemas de programación complejos \citep{GAschedulingSurvey}. Esta estrategia permite al sistema encontrar soluciones de alta calidad (casi óptimas) dentro de un límite de tiempo fijo, lo cual resulta aceptable y práctico para las necesidades de la operación real.

\section{Objetivos y alcance}

Basado en la problemática y la motivación expuestas, el objetivo general de este trabajo es diseñar, implementar y validar el software que constituye la ``Capa de Gestión y Optimización`` para un sistema de riego solar. El proyecto se enfoca en desarrollar los dos componentes técnicos fundamentales que responden directamente a los dilemas identificados en el estado del arte.

El primer objetivo específico es la creación de un Gemelo Digital Híbrido. Esto implica implementar un simulador físico (basado en EPANET/WNTR) y, de forma crucial, desarrollar un modelo de inteligencia artificial cuya función es predecir y corregir el error residual de dicho modelo físico. Este componente servirá como la función de evaluación precisa y robusta para el optimizador.

El segundo objetivo específico es el diseño de un optimizador metaheurístico. Se implementará un Algoritmo Genético (GA) capaz de navegar el complejo espacio de búsqueda MINLP . Este agente utilizará el Gemelo Digital Híbrido como su función de aptitud para generar cronogramas de despacho casi óptimos, priorizando la factibilidad temporal sobre la optimalidad exacta. El alcance de la implementación incluye también el desarrollo de la función de costos híbrida y unificada, que permite al agrónomo ponderar los objetivos de uniformidad, eficiencia y sostenibilidad del acuífero.

Es importante delimitar que el alcance de este trabajo es exclusivamente de software y se centra en el motor de decisión. El proyecto no incluye el diseño, provisión, instalación o mantenimiento de ningún componente de hardware (como bombas, sensores, PLCs o gateways). Asimismo, el desarrollo de la interfaz de usuario final o ``Capa de Presentación`` (dashboard) y el soporte operativo post-entrega quedan fuera del alcance de esta memoria. Esta arquitectura global, y el foco específico del proyecto, se ilustran en la figura \ref{fig:diagrama_sistema}.

\begin{figure}[ht]
	\centering
	\includegraphics[width=\textwidth]{Figures/despacho_v1.pdf}
	\caption{Diagrama en bloques de la arquitectura completa del sistema de riego.}
	\label{fig:diagrama_sistema}
\end{figure}