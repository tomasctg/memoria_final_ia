\chapter{Introducción específica}
\label{chap:introduccion_especifica}

En este capítulo se presentan los fundamentos teóricos y las herramientas que sustentan la arquitectura de optimización propuesta. Se detalla el modelado híbrido de caja gris, la formulación del problema MINLP, la configuración del algoritmo genético y las métricas de evaluación empleadas.

\section{Teoría del modelado híbrido: corrección de residuales}
\label{sec:teoria_modelado_hibrido}

La simulación de sistemas hidráulicos presenta un compromiso entre la consistencia física y la precisión operativa. Mientras que los modelos de caja blanca (determinísticos) como EPANET garantizan coherencia física pero sufren sesgos por incertidumbre paramétrica, los modelos de caja negra (estocásticos) ofrecen alta precisión pero carecen de robustez física \citep{ref7, ref4}.

Para balancear estas limitaciones, este trabajo implementa un modelado de caja gris (\textit{gray-box}) mediante una arquitectura de corrección de residuales \citep{ref5}. En este esquema, el modelo físico ($f_{fisico}$) aporta la línea base robusta, mientras que un modelo de inteligencia artificial ($f_{AI}$) predice y corrige el error sistemático o residual ($R$) del primero.

La formulación matemática de esta arquitectura híbrida se define en la ecuación \ref{f_hibrido}:
\begin{equation}\label{f_hibrido}
f_{hybrid} = f_{fisico} + f_{AI}(\mathbf{X})
\end{equation}
donde el objetivo de entrenamiento para $f_{AI}$ es el residual $R = Y_{real} - Y_{fisico}$.

\subsection{Componente físico ($f_{fisico}$)}
\label{subsec:componente_fisico}

El componente físico se desarrolla utilizando el motor de simulación EPANET \citep{ref9, ref10}. Este modelo se rige por leyes físicas claves que definen la complejidad del problema:

\begin{enumerate}
    \item Pérdida de carga por fricción: se utiliza la ecuación de Hazen-Williams (ecuación \ref{HW}) \citep{meschter_2001_hazen}, una relación empírica que define la pérdida de carga ($H_f$) en una tubería. 
    \begin{equation}\label{HW}
    H_f = \frac{10.67 \cdot L}{C^{1.852} \cdot D^{4.87}} \cdot Q^{1.852}
    \end{equation}
    
    \item Comportamiento de emisores (goteros): el caudal de salida ($q_e$) se modela mediante la ecuación caudal-presión del emisor \citep{keller_karmeli_1975}, una relación empírica de tipo potencia (ecuación \ref{ley_potencia}) que vincula el caudal con la presión ($p$) en el nodo. En este trabajo se utiliza un exponente $\gamma$ de 0,46, característico del equipamiento instalado, introduciendo otra no linealidad.
    \begin{equation}\label{ley_potencia}
    q_e = C_d \cdot p^{\gamma}
    \end{equation}
\end{enumerate}

\subsection{Componente corrector ($f_{AI}$)}
El componente de corrección se implementa mediante un perceptrón multicapa (MLP). Este recibe como entrada el estado operativo del sistema (presión de bomba, nivel freático y configuración de válvulas) junto con la salida del modelo físico, aprendiendo a estimar la diferencia entre la simulación teórica y los datos reales medidos.

\section{Formulación del problema de optimización}
\label{sec:formulacion_optimizacion}

Una vez que se dispone de una función de aptitud $f_{hybrid}$ de alta fidelidad, es necesario definir formalmente el problema de optimización.

\subsection{Formulación matemática formal}
\label{subsec:formulacion_formal}

El problema de optimización del despacho de riego consiste en encontrar el cronograma de apertura y cierre de válvulas que maximice un conjunto de objetivos, sujeto a las restricciones físicas de la red hidráulica y la disponibilidad de recursos. En tabla \ref{tab:nomenclatura}, se presenta una nomenclatura y formulación general del modelo.

\begin{table}[h!]
\centering
\caption{Nomenclatura del modelo de optimización. }
\label{tab:nomenclatura}
\begin{tabular}{l p{10cm}}
\toprule
\textbf{Símbolo} & \textbf{Definición} \\
\midrule
$\mathbf{V}$ & Matriz de decisiones de despacho (variable de decisión). \\
$v_{i,t}$ & Variable binaria: 1 si la válvula $i$ está abierta en el tiempo $t$, 0 si está cerrada. \\
$\mathcal{T}$ & Horizonte de planificación (ej. 24 horas, discretizado en $T$ intervalos). \\
$\mathcal{I}$ & Conjunto de todas las válvulas (sectores de riego). \\
$f_{hybrid}(\mathbf{V})$ & Función de aptitud. \\
$Q_{entregado,i}(\mathbf{V})$ & Caudal total entregado al sector $i$ bajo el plan $\mathbf{V}$. \\
$Q_{objetivo,i}$ & Caudal objetivo para el sector $i$. \\
$P_{consumida,t}(\mathbf{V})$ & Potencia consumida por la bomba en el tiempo $t$ bajo el plan $\mathbf{V}$. \\
$P_{disponible,t}$ & Potencia disponible pronosticada en el tiempo $t$. \\
$N_{recurso,t}(\mathbf{V})$ & Nivel del recurso en el tiempo $t$ bajo el plan $\mathbf{V}$. \\
$N_{min}$ & Nivel mínimo de seguridad del recurso. \\
$w_u, w_t, w_a$ & Ponderaciones estratégicas para los objetivos de la función. \\
\bottomrule
\end{tabular}
\end{table}

\subsection{Función objetivo ponderada}

Un enfoque común para problemas multiobjetivo es formular una función de aptitud unificada. Esta función es una suma ponderada que puede incluir objetivos agronómicos, de eficiencia y de sostenibilidad. Esta se muestra en la ecuación \ref{fun_costo}:

\begin{equation}\label{fun_costo}
\max_{\mathbf{V}} \left( w_u \cdot \mathcal{U}(\mathbf{V}) - w_t \cdot \mathcal{T}_{bombeo}(\mathbf{V}) - w_a \cdot \mathcal{S}_{acuifero}(\mathbf{V}) \right)
\end{equation}

donde:
\begin{itemize}
    \item $\mathcal{U}(\mathbf{V})$: una métrica de uniformidad, que se busca maximizar.
    \item $\mathcal{T}_{bombeo}(\mathbf{V})$: una métrica del tiempo de bombeo, que se busca minimizar.
    \item $\mathcal{S}_{acuifero}(\mathbf{V})$: una métrica del estrés sobre el acuífero, que se busca minimizar.
\end{itemize}

\subsection{Restricciones del problema}

La maximización está sujeta a un conjunto de restricciones físicas y operativas:

\begin{enumerate}
    \item Restricción de potencia: la potencia consumida por la bomba no puede exceder la potencia disponible en ningún momento, de acuerdo a la ecuación \ref{rest_potencia}.
    \begin{equation}\label{rest_potencia}
    P_{consumida,t}(f_{hybrid}(\mathbf{V})) \le P_{disponible,t} \quad \forall t \in \mathcal{T}
    \end{equation}
    
    \item Restricción de sostenibilidad (ej. nivel freático): el nivel del recurso no debe caer por debajo del límite de seguridad, según la ecuación \ref{nivel_frea}.
    \begin{equation}\label{nivel_frea}
    N_{recurso,t}(f_{hybrid}(\mathbf{V})) \ge N_{min} \quad \forall t \in \mathcal{T}
    \end{equation}
    
    \item Restricciones hidráulicas (implícitas): las leyes físicas de la red están encapsuladas dentro de la evaluación de la función $f_{hybrid}$ .
    
    \item Variables de decisión binarias: las decisiones de control de válvulas son binarias, como lo indica la ecuación \ref{valv_bin}.
    \begin{equation}\label{valv_bin}
    v_{i,t} \in \{0, 1\} \quad \forall i \in \mathcal{I}, \forall t \in \mathcal{T}
    \end{equation}
\end{enumerate}

\section{Teoría de optimización: algoritmos genéticos}
\label{sec:teoria_AG}

El algoritmo genético consiste en una búsqueda estocástica global basada en los principios de la evolución natural y la supervivencia del más apto \citep{ref15, ref16}. Este algoritmo es adecuado para navegar espacios de búsqueda complejos, no convexos y de alta dimensionalidad. Su uso acoplado a simuladores como EPANET para la optimización de redes de agua es una técnica documentada en la literatura \citep{ref9, ref10}.

El AG opera sobre una población de cromosomas, donde cada cromosoma representa una solución candidata completa al problema de optimización.

\subsection{Codificación del cromosoma (representación de la solución)}
\label{subsubsec:codificacion_cromosoma}

El cromosoma es la estructura de datos que codifica un plan de despacho de riego \citep{ref17}. La elección de la codificación es fundamental para el rendimiento del algoritmo.

Se utiliza una codificación por valor binario (\textit{value encoding}). El cromosoma es una cadena de valores (enteros, reales o binarios) donde cada gen representa un valor específico. Si el horizonte de planificación $T$ se discretiza en $N_T$ intervalos y hay $N_V$ válvulas, un cromosoma $C$ es un vector binario de longitud $N_V \times N_T$. Aunque se almacena como un vector, conceptualmente representa una matriz donde cada elemento $v_{i,t}$ es un gen, como muestra la ecuación \ref{eq:cromosoma}:
\begin{equation}
\label{eq:cromosoma}
C = 
\begin{bmatrix}
v_{1,1} & v_{1,2} & \dots  & v_{1,N_T} \\
v_{2,1} & v_{2,2} & \dots  & v_{2,N_T} \\
\vdots  & \vdots  & \ddots & \vdots    \\
v_{N_V,1} & v_{N_V,2} & \dots & v_{N_V,N_T}
\end{bmatrix}
\end{equation}
donde $v_{i,t} \in \{0, 1\}$ representa el estado (cerrado/abierto) de la válvula $i$ en el tiempo $t$.

\subsection{Operadores evolutivos}
Para garantizar la convergencia y diversidad de la población, se consideran los siguientes operadores:

\begin{itemize}
    \item Selección por torneo: se eligen $k$ individuos al azar y se selecciona el de mejor aptitud \citep{ref20}.
    
    \item Cruce uniforme: dado que el cromosoma es una matriz binaria sin dependencia posicional estricta, el cruce uniforme permite una recombinación eficiente de los genes. Cada bit del hijo se selecciona probabilísticamente de uno de los dos padres, promoviendo una mayor exploración del espacio de búsqueda \citep{ref19}.
    
    \item Mutación \textit{bit-flip}: se aplica una mutación puntual donde un gen (estado de válvula) invierte su valor ($0 \to 1$ o $1 \to 0$) con una baja probabilidad. Esto introduce diversidad genética y previene el estancamiento en óptimos locales.
\end{itemize}

\section{Herramientas computacionales y métricas}
\label{sec:herramientas_metricas}

Para la materialización de los conceptos teóricos expuestos y la validación de los algoritmos de optimización, se selecciona un conjunto específico de herramientas de \textit{software} y métricas de desempeño.

\subsection{Bibliotecas y entorno de desarrollo}
El desarrollo se fundamenta en el lenguaje de programación \texttt{Python}, seleccionado por su extenso ecosistema para la ciencia de datos y la ingeniería. Las bibliotecas principales se categorizan según su función en la arquitectura:

\begin{itemize}
    \item Simulación hidráulica:
    para la implementación del componente físico ($f_{fisico}$), se utiliza la biblioteca \textit{Water Network Tool for Resilience} (WNTR) \citep{WNTRShaw2018}. Esta herramienta, basada en el motor estándar EPANET, permite la simulación hidráulica dinámica, facilitando la inyección de parámetros de control en tiempo de ejecución.

    \item Aprendizaje profundo: 
    el componente corrector de residuales ($f_{AI}$) se construye sobre PyTorch \citep{Paszke2019_PyTorch}. Esta biblioteca ofrece la flexibilidad necesaria para diseñar arquitecturas de redes neuronales personalizadas y aprovecha la aceleración por hardware mediante tensores, lo cual es crítico para minimizar la latencia durante las miles de evaluaciones requeridas por el AG.

    \item Procesamiento numérico y datos:
    la manipulación eficiente de los cromosomas (matrices binarias) y las operaciones vectoriales del AG se realizan mediante NumPy, asegurando un bajo costo computacional en los operadores de cruce y mutación. Por su parte, Pandas se emplea para la gestión de series temporales y la ingeniería de características previa al entrenamiento.

    \item Gestión del ciclo de vida: 
    para garantizar la reproducibilidad científica del modelo híbrido, se utiliza MLflow. Esta plataforma permite registrar sistemáticamente los hiperparámetros del modelo, las versiones de los datos de entrenamiento y las métricas de validación, asegurando la trazabilidad completa del experimento.
\end{itemize}

\subsection{Métricas de evaluación}
La evaluación del desempeño del sistema se aborda desde dos perspectivas complementarias: la precisión predictiva del gemelo digital híbrido y la eficacia del motor de búsqueda del AG.

Para cuantificar la capacidad del modelo híbrido ($f_{hybrid}$) de replicar la realidad y corregir el sesgo del modelo físico, se emplea principalmente la raíz del error cuadrático medio (RMSE) como se observa en la ecuación \ref{ecm}. 

\begin{equation}\label{ecm}
RMSE = \sqrt{\frac{1}{n} \sum_{i=1}^{n} (y_{i} - \hat{y}_{i})^2}
\end{equation}

donde $y_i$ representa el valor real medido y $\hat{y}_i$ la estimación del modelo. Adicionalmente, se utiliza el coeficiente de determinación ($R^2$) para medir la proporción de la varianza de los datos reales que es explicada por el modelo, validando su capacidad para capturar la dinámica del sistema.

Por otro lado, la calidad del proceso de optimización se evalúa mediante el monitoreo del costo de la mejor solución a lo largo de las generaciones, lo que permite verificar la convergencia del algoritmo. 





