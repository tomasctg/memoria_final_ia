\chapter{Introducción específica}
\label{chap:introduccion_especifica}

En este capítulo se presentan los fundamentos teóricos que sustentan la arquitectura de optimización del proyecto. Se detalla el modelado híbrido de ''caja gris'', la formulación matemática del problema como un MINLP, la teoría de los Algoritmos Genéticos como solución metaheurística y el Método de Funciones de Penalización para el manejo de restricciones.

\section{Teoría del modelado híbrido: corrección de residuales}
\label{sec:teoria_modelado_hibrido}

La simulación de sistemas físicos complejos, como lo es un sistema hidráulico, se aborda generalmente desde dos enfoques principales \citep{ref7}.

El primer enfoque es el modelado de ''Caja blanca'' (\textit{White-Box}), de naturaleza determinística, basado en leyes físicas de la mecánica de fluidos. \citep{ref7, ref8}. El motor de simulación hidráulico \texttt{EPANET} es un ejemplo representativo de este enfoque \citep{ref9, ref10}. La ventaja principal de los modelos de caja blanca es su interpretabilidad y la garantía de predicciones físicamente consistentes. Sin embargo, los modelos que se basan únicamente en leyes físicas tienen una limitación fundamental: debido a simplificaciones, inevitablemente producen desviaciones significativas, o sesgos, en comparación con la realidad \citep{ref4, ref5}. Estos sesgos surgen de dinámicas no modeladas y de la incertidumbre en los parámetros intrínsecos, como la rugosidad exacta de las tuberías (el coeficiente $C$ en la ecuación de Hazen-Williams) o las pérdidas de carga menores no contempladas en el modelo .

El segundo enfoque es el modelado de ''Caja negra'' (\textit{Black-Box}). Estos modelos son de naturaleza estocástica, como las redes neuronales (NN) o los perceptrones multicapa (MLP), que mapean entradas a salidas basándose puramente en datos observados \citep{ref7, ref8}. Su fortaleza es la capacidad de modelar relaciones no lineales extremadamente complejas sin necesidad de conocer el modelo físico del sistema \citep{ref6}. Sin embargo, sus desventajas son críticas para sistemas de ingeniería. Primero, requieren volúmenes de datos de entrenamiento que a menudo son muy grandes \citep{ref4, ref5}. Segundo, y más importante, son propensos a generar predicciones que pueden ser físicamente inconsistentes, como violar la conservación de masa, un riesgo inaceptable para un sistema de control de recursos hídricos .

Como solución, en este proyecto se propone un modelado de ''Caja Gris'' (\textit{Gray-Box}), un paradigma que busca combinar las fortalezas de ambos enfoques\citep{ref6}. Específicamente, se implementa la arquitectura de corrección de residuales .

En este paradigma, el modelo de inteligencia artificial no intenta aprender la totalidad del sistema desde cero. En su lugar, el modelo físico ($f_{fisico}$) proporciona la predicción principal y la robustez, mientras que el modelo de IA ($f_{AI}$) tiene una tarea mucho más acotada y factible: aprender y predecir el error residual (el sesgo sistemático) del modelo físico \citep{ref5}.

La formulación matemática de esta arquitectura híbrida es :
\begin{equation}\label{f_hibrido}
f_{hybrid} = f_{fisico} + f_{AI}(\mathbf{X})
\end{equation}
Donde el objetivo de entrenamiento (el \textit{target}) para $f_{AI}$ es el residual $R$, definido como :
\begin{equation}
R = Y_{real} - Y_{fisico}
\end{equation}
La literatura académica lo describe como una herramienta de predicción eficiente \citep{ref4, ref5}, donde el modelo $f_{fisico}$ aporta la consistencia y el $f_{AI}$ aporta la precisión, corrigiendo los sesgos propios del primero .

A continuación se analizan ambas partes de la ecuación \ref{f_hibrido}.

\subsection{Componente físico ($f_{fisico}$)}
\label{subsec:componente_fisico}

El componente físico se desarrolla utilizando el motor de simulación \texttt{EPANET} \citep{ref9, ref10}. Este modelo se rige por leyes físicas claves que definen la complejidad del problema :

\begin{enumerate}
    \item Pérdida de carga por fricción: se utiliza la ecuación de Hazen-Williams, una relación empírica que define la pérdida de carga ($H_f$) en una tubería. 
    \begin{equation}
    H_f = \frac{10.67 \cdot L}{C^{1.852} \cdot D^{4.87}} \cdot Q^{1.852}
    \end{equation}
    
    \item Comportamiento de emisores (goteros): el caudal de salida ($q_e$) en un gotero se modela usando la ley de potencia, que relaciona el caudal con la presión ($p$) en el nodo. En el proyecto se utiliza un exponente $\gamma$ de 0.46, característico del equipamiento instalado, introduciendo otra no linealidad.
    .
    \begin{equation}
    q_e = C_d \cdot p^{\gamma}
    \end{equation}
\end{enumerate}

Además, se emplea la simulación en modo Análisis Dirigido por Presión (PDA). A diferencia del Análisis Dirigido por Demanda (DDA), que asume que la demanda siempre se satisface, el PDA ajusta el caudal entregado si la presión en un nodo cae, ofreciendo una simulación mucho más realista para un sistema de riego solar donde la presión de la bomba varía constantemente.

\subsection{Componente corrector ($f_{AI}$)}
\label{subsec:componente_corrector}

El componente de IA, $f_{AI}$, se implementa como un Perceptrón Multicapa (MLP).

El vector de características $\mathbf{X}$ incluye estados operativos reales (como la presión y nivel freatico del aquifero) y decisiones de control (como el estado de las válvulas). También incluye la propia salida del modelo físico como una característica de entrada. Esto significa que el MLP aprende explícitamente en función de la predicción física, operando como un verdadero corrector de sesgos .

\section{Formulación del problema de optimización}
\label{sec:formulacion_optimizacion}

Una vez que se dispone de una función de aptitud $f_{hybrid}$ de alta fidelidad,es necesario definir formalmente el problema de optimización.

\subsection{Formulación matemática formal}
\label{subsec:formulacion_formal}

El problema de optimización del despacho de riego consiste en encontrar el cronograma de apertura y cierre de válvulas que maximice un conjunto de objetivos, sujeto a las restricciones físicas de la red hidráulica y la disponibilidad de recursos. A continuación, se presenta una nomenclatura y formulación general del modelo .

\begin{table}[h!]
\centering
\caption{Nomenclatura del Modelo de Optimización. }
\label{tab:nomenclatura}
\begin{tabular}{l p{10cm}}
\toprule
\textbf{Símbolo} & \textbf{Definición} \\
\midrule
$\mathbf{V}$ & Matriz de decisiones de despacho (variable de decisión). \\
$v_{i,t}$ & Variable binaria: 1 si la válvula $i$ está abierta en el tiempo $t$, 0 si está cerrada. \\
$\mathcal{T}$ & Horizonte de planificación (ej. 24 horas, discretizado en $T$ intervalos). \\
$\mathcal{I}$ & Conjunto de todas las válvulas (sectores de riego). \\
$f_{hybrid}(\mathbf{V})$ & Función de aptitud (el gemelo digital) que mapea un plan $\mathbf{V}$ a sus consecuencias físicas. \\
$Q_{entregado,i}(\mathbf{V})$ & Caudal total entregado al sector $i$ bajo el plan $\mathbf{V}$. \\
$Q_{objetivo,i}$ & Caudal objetivo para el sector $i$. \\
$P_{consumida,t}(\mathbf{V})$ & Potencia consumida por la bomba en el tiempo $t$ bajo el plan $\mathbf{V}$. \\
$P_{disponible,t}$ & Potencia disponible pronosticada en el tiempo $t$ (ej. solar). \\
$N_{recurso,t}(\mathbf{V})$ & Nivel del recurso (ej. freático) en el tiempo $t$ bajo el plan $\mathbf{V}$. \\
$N_{min}$ & Nivel mínimo de seguridad del recurso. \\
$w_u, w_t, w_a$ & Ponderaciones estratégicas para los objetivos de la función. \\
\bottomrule
\end{tabular}
\end{table}

\subsubsection{Función objetivo ponderada}

Un enfoque común para problemas multiobjetivo es formular una función de aptitud unificada. Esta función es una suma ponderada que puede incluir objetivos agronómicos, de eficiencia y de sostenibilidad :

\begin{equation}
\max_{\mathbf{V}} \left( w_u \cdot \mathcal{U}(\mathbf{V}) - w_t \cdot \mathcal{T}_{bombeo}(\mathbf{V}) - w_a \cdot \mathcal{S}_{acuifero}(\mathbf{V}) \right)
\end{equation}

Donde :
\begin{itemize}
    \item $\mathcal{U}(\mathbf{V})$: Una métrica de uniformidad (ej. minimizar el déficit cuadrático $\sum (Q_{objetivo,i} - Q_{entregado,i})^2$), que se busca maximizar.
    \item $\mathcal{T}_{bombeo}(\mathbf{V})$: Una métrica del tiempo de bombeo (ej. $\sum v_{i,t}$), que se busca minimizar.
    \item $\mathcal{S}_{acuifero}(\mathbf{V})$: Una métrica del estrés sobre el acuífero (ej. la extracción total de agua), que se busca minimizar.
\end{itemize}

\subsubsection{Restricciones del problema}

La maximización está sujeta a un conjunto de restricciones físicas y operativas :

\begin{enumerate}
    \item Restricción de potencia: la potencia consumida por la bomba no puede exceder la potencia disponible en ningún momento.
    \begin{equation}
    P_{consumida,t}(f_{hybrid}(\mathbf{V})) \le P_{disponible,t} \quad \forall t \in \mathcal{T}
    \end{equation}
    
    \item Restricción de sostenibilidad (ej. nivel freático): el nivel del recurso no debe caer por debajo del límite de seguridad.
    \begin{equation}
    N_{recurso,t}(f_{hybrid}(\mathbf{V})) \ge N_{min} \quad \forall t \in \mathcal{T}
    \end{equation}
    
    \item Restricciones hidráulicas (implícitas): las leyes físicas de la red (Hazen-Williams, Ley de Potencia) están encapsuladas dentro de la evaluación de la función $f_{hybrid}$ .
    
    \item Variables de decisión binarias: las decisiones de control de válvulas son binarias.
    \begin{equation}
    v_{i,t} \in \{0, 1\} \quad \forall i \in \mathcal{I}, \forall t \in \mathcal{T}
    \end{equation}
\end{enumerate}


