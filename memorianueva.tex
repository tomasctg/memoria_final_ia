%----------------------------------------------------------------------------------------
% Preambulo y Configuración
%----------------------------------------------------------------------------------------

\documentclass[
    11pt,
    spanish,
    singlespacing,
    parskip,
    headsepline,
    bookmarks=true,
    unicode=true,
    pdftoolbar=true,
    pdfmenubar=true,
    pdffitwindow=false,
    colorlinks=true,
    linkcolor=blue,
    citecolor=blue,
    urlcolor=blue
]{MastersDoctoralThesis}

\usepackage[utf8]{inputenc} % Codificación de entrada UTF-8
\usepackage[T1]{fontenc}    % Codificación de salida para caracteres especiales
\usepackage{graphicx}       % Manejo de gráficos
\usepackage{eso-pic}        % Permite agregar fondos
\usepackage{hyperref}       % Manejo de hipervínculos y marcadores
\usepackage{amsmath}

% Redefinición de caracteres problemáticos en marcadores
\hypersetup{
    pdftitle={Título del Documento},
    pdfauthor={Autor del Documento},
    pdfkeywords={Sistemas Embebidos, Internet de las Cosas, Inteligencia Artificial},
    pdfstartview={FitH},
    unicode=true,
    colorlinks=true,
    linkcolor=blue,
    citecolor=blue,
    urlcolor=blue
}

\pdfstringdefDisableCommands{%
  \def\texttt#1{#1}%
  \def\textbf#1{#1}%
  \def\textit#1{#1}%
  \def\"{\"}%
  \def\~{~}%
  \def\'{'}%
  \def\^{}%
  \def\textunderscore{\_} % Manejo del subrayado en marcadores
}


% Definir comandos requeridos por la clase
\newcommand{\degreename}{Maestría en Ciencias} % Cambia según tu título
\newcommand{\univname}{Universidad Nacional de Ejemplo} % Cambia según tu universidad
\newcommand{\keywordnames}{Palabras clave:}
%----------------------------------------------------------------------------------------
% Documento Principal
%----------------------------------------------------------------------------------------

\begin{document}

% Configuración de la portada
\posgrado{Carrera / Maestría}
\keywords{Sistemas Embebidos, Internet de las Cosas, Inteligencia Artificial}

% Incluir la portada desde un archivo separado
%----------------------------------------------------------------------------------------
% PORTADA
%----------------------------------------------------------------------------------------
\begin{titlepage}
    % Fondo completo con el PDF que incluye la barra y el logo
    \AddToShipoutPictureBG*{\includegraphics[width=\paperwidth, height=\paperheight]{Figures/fondo.pdf}}

    % Contenido principal
    \begin{flushright}
        \setlength{\rightskip}{-2cm} % Ajusta la sangría derecha
        \vspace*{7.5cm} % Ajustar según la posición vertical deseada

        % Título
        {\fontfamily{phv}\bfseries\fontsize{33pt}{40pt}\selectfont
        Optimización con inteligencia artificial del despacho de riego solar en cultivos de pistacho} \\[1.5cm]

        % Autor
        {\fontfamily{phv}\fontsize{20pt}{25pt}\selectfont
        Ing. Tomás Corteggiano} \\[1cm]

        % Carrera o Maestría (comentar o descomentar la línea correspondiente)
        {\fontfamily{phv}\fontsize{15pt}{20pt}\selectfont
        \textbf{Carrera de Especialización en Inteligencia Artificial} 
        } \\[2cm]

        % Director
        {\fontfamily{phv}\fontsize{11pt}{15pt}\selectfont
        \textbf{Director:} Dr. Ing. Carlos Larisson} \\[1cm]

        % Jurados
        {\fontfamily{phv}\fontsize{11pt}{15pt}\selectfont
        \textbf{Jurados:}} \\[0.5cm]
        {\fontfamily{phv}\fontsize{11pt}{15pt}\selectfont
        Jurado 1 (pertenencia)} \\ 
        {\fontfamily{phv}\fontsize{11pt}{15pt}\selectfont
        Jurado 2 (pertenencia)} \\ 
        {\fontfamily{phv}\fontsize{11pt}{15pt}\selectfont
        Jurado 3 (pertenencia)} \\[1cm]

        % Fecha y lugar
        {\fontfamily{phv}\itshape\fontsize{10pt}{12pt}\selectfont
        Ciudad de Mendoza, noviembre de 2025} % Ejemplo: Ciudad de Córdoba, junio de 2025
    \end{flushright}
\end{titlepage}


% Configuración del contenido preliminar
\frontmatter % Usar numeración romana para las páginas preliminares
\pagestyle{plain} % Estilo de encabezado simple

%----------------------------------------------------------------------------------------
% Resumen
%----------------------------------------------------------------------------------------

\begin{abstract}
\addchaptertocentry{\abstractname} % Agregar resumen al índice
La presente memoria describe el desarrollo de un sistema avanzado para la optimización del riego en una plantación de pistachos, con el objetivo de generar un cronograma de despacho de riego óptimo para una empresa agrícola. El sistema busca balancear objetivos clave como la uniformidad del riego, la eficiencia del bombeo y la sostenibilidad del acuífero mediante una función de costos unificada.

Para superar las limitaciones de los modelos físicos, se creó un gemelo digital híbrido de alta precisión combinando modelado físico con técnicas de Inteligencia Artificial (IA) para corregir errores residuales. Además, se utilizó un Algoritmo Genético, un algoritmo de búsqueda avanzado, para resolver eficientemente el complejo Problema de Programación Mixta-Entera No Lineal (MINLP) asociado al despacho de riego.

\end{abstract}


%----------------------------------------------------------------------------------------
% Índice
%----------------------------------------------------------------------------------------

\tableofcontents
\listoffigures
\listoftables


%----------------------------------------------------------------------------------------
% Capítulos
%----------------------------------------------------------------------------------------

\mainmatter % Iniciar numeración numérica para el contenido principal
\pagestyle{thesis} % Estilo de encabezado de tesis

% Incluir capítulos desde archivos separados
% Chapter 1

\chapter{Introducción} % Main chapter title

\label{Chapter1} % For referencing the chapter elsewhere, use \ref{Chapter1} 
\label{IntroGeneral}

En este capítulo se presenta una introducción general al proyecto. Se describe la problemática fundamental del riego solar que motivó este trabajo y se analiza el estado del arte en las áreas clave del modelado de sistemas y su optimización. Finalmente, se expone la motivación de la solución propuesta y se delimitan los objetivos y el alcance del software que fue desarrollado.

%----------------------------------------------------------------------------------------

% Define some commands to keep the formatting separated from the content 
\newcommand{\keyword}[1]{\textbf{#1}}
\newcommand{\tabhead}[1]{\textbf{#1}}
\newcommand{\code}[1]{\texttt{#1}}
\newcommand{\file}[1]{\texttt{\bfseries#1}}
\newcommand{\option}[1]{\texttt{\itshape#1}}
\newcommand{\grados}{$^{\circ}$}

%----------------------------------------------------------------------------------------

%\section{Introducción}

%----------------------------------------------------------------------------------------
\section{Contexto y problemática del riego solar}

La implementación de sistemas de riego 100\% solares es una solución tecnológica de amplio uso en el sector agrícola, particularmente en plantaciones que operan de forma autónoma (es decir, en modo \textit{off-grid}), donde la red eléctrica de distribución pública no está disponible o su conexión resulta económicamente inviable. Estas instalaciones se componen típicamente de una \textbf{Capa Física} (la bomba sumergible y las válvulas de campo) y una \textbf{Capa de Control y Adquisición} (el variador de frecuencia (VFD) y el PLC que comandan dicho hardware). En la operación estándar, esta arquitectura de control es robusta para ejecutar comandos, pero carece de la capa superior de optimización que define el plan de operación. 

Esta ausencia de una capa de gestión inteligente traslada por completo la carga de optimización al operario. El despacho de agua se convierte en una consecuencia pasiva de la meteorología —siguiendo la curva de irradiación solar— o, en el mejor de los casos, sigue un plan estático (basado en temporizadores) definido manualmente por el ingeniero agrónomo. Este enfoque manual representa una carga significativa de horas de ingeniería y es inherentemente subóptimo, ya que no puede reaccionar a la variabilidad diaria de las condiciones hídricas o energéticas. 

La necesidad de esta optimización inteligente se vuelve crítica en contextos agrícolas como el del cultivo de pistacho, donde los ciclos de inversión son largos y la planta requiere varios años para entrar en producción. En regiones donde el agua es un recurso escaso o sujetas a sequías estacionales, una gestión deficiente del riego no solo reduce la eficiencia, sino que compromete la viabilidad y sostenibilidad a largo plazo de la plantación. 

Por lo tanto, la problemática central que este trabajo aborda es esta falta de inteligencia operativa. La proliferación de la telemetría y los sensores de campo (IoT) en la agricultura moderna han generado un vasto potencial de optimización que, en la mayoría de los sistemas de riego, permanece sin explotar \cite{IoTReviewMDPI}. Los controladores actuales se limitan a registrar estos datos o a reaccionar a umbrales simples, pero carecen de la capacidad de utilizarlos para una planificación proactiva.

El verdadero desafío que se resuelve en este trabajo es integrar la Inteligencia Artificial para dos tareas claves: primero, para transformar este flujo de datos en conocimiento predictivo (un modelo de alta fidelidad del sistema); y segundo, para actuar sobre dicho conocimiento. Se requiere un motor de optimización —en este caso, un Algoritmo Genético— que funcione como un buscador capaz de \textit{generar} un cronograma de despacho óptimo a largo plazo. Esta capacidad de búsqueda y planificación proactiva, que balancea múltiples objetivos, es la pieza de gestión que falta en los sistemas actuales, los cuales obligan a operar muy por debajo de su potencial de eficiencia. 

%----------------------------------------------------------------------------------------

\section{Estado del arte: modelado y optimización}

El problema central de este trabajo —generar un cronograma de despacho de riego óptimo para un sistema solar autónomo— no es un desafío singular, sino la intersección de dos conceptos técnicos que el estado del arte ha abordado históricamente por vías separadas:
\begin{enumerate}
    \item \textbf{El desafío del modelado:} se requiere un simulador de muy alta fidelidad, un ``gemelo digital``, que actúe como la función de evaluación del optimizador. Este modelo debe predecir con precisión la respuesta hidráulica (caudales, presiones, nivel freático) ante cualquier plan de riego candidato. 
    \item \textbf{El desafío de la optimización:} se necesita un algoritmo de búsqueda capaz de encontrar la combinación óptima de decisiones (qué válvula abrir y cuándo) dentro de un espacio de soluciones de dimensionalidad masiva, respetando un conjunto de restricciones físicas (energía, agua, presión) y un tiempo de cómputo estricto. 
\end{enumerate}

A continuación, se analiza el panorama de las soluciones existentes para cada uno de estos desafíos, identificando las limitaciones que motivan el desarrollo de este trabajo. 

\subsection{El modelado: consistencia vs. precisión }

El algoritmo de optimización es, por naturaleza, agnóstico a la física del problema; depende estrictamente de una \textbf{función de evaluación} (en este caso, el gemelo digital) para cuantificar la calidad, o \textit{fitness}, de un cronograma de riego candidato. El éxito de la optimización, por lo tanto, depende enteramente de la fidelidad de este modelo. El enfoque tradicional para construir esta función se basa en modelos de \textbf{``caja blanca``} o de primeros principios, como el estándar industrial EPANET, que resuelven las ecuaciones fundamentales de la mecánica de fluidos \cite{DITEC-WNTR, WNTR-Python}. Su fortaleza indiscutible es la \textbf{consistencia física}: al estar basados en leyes universales, son robustos y pueden generalizar su comportamiento a escenarios de operación novedosos \cite{HybridPatternsArxiv}. Sin embargo, su debilidad es la \textbf{incertidumbre }en sus parámetros físicos; el rendimiento del modelo depende de una calibración precisa de variables estáticas (ej. rugosidad de tuberías) que son difíciles de medir y cambian con el tiempo \cite{CalibrationModelANN}. Esto genera un \textbf{error sistemático o ``residual``} entre la simulación idealizada y la realidad del campo. 

Para solucionar esta brecha de precisión, en el extremo opuesto del espectro se encuentran los modelos puramente empíricos de \textbf{``caja negra``} (IA), que buscan aprender el comportamiento del sistema basándose únicamente en datos históricos de telemetría. La ventaja de este enfoque es la capacidad de alcanzar una alta precisión, capturando dinámicas complejas que el modelo físico ignora \cite{HybridReviewTandF}. No obstante, esta solución es \textbf{frágil}: requiere volúmenes masivos de datos para cubrir todo el espacio operativo \cite{ReviewPhysicsInformed}, no es robusta para extrapolar a escenarios no vistos y, crucialmente, no garantiza la consistencia física \cite{ReviewPhysicsInformed}.  

El panorama de las soluciones de modelado presenta, por lo tanto, un dilema fundamental: el ingeniero debe elegir entre la \textbf{consistencia física} de la ``caja blanca`` (sufriendo su imprecisión) o la \textbf{precisión} de la ``caja negra`` (sufriendo su falta de robustez). Esta dicotomía, junto con la alternativa conceptual de un modelo híbrido o de ``caja gris`` que busca unificar ambas \citep{GreyBoxWikipedia, GreyBoxHydrology}, se ilustra en la figura \ref{fig:diagrama_caja_negra_blanca}. 

\begin{figure}[ht]
	\centering
	\includegraphics[width=\textwidth]{Figures/diagrama_cajas-2.pdf}
	\caption{Diagrama conceptual de los paradigmas de modelado: caja blanca (física), caja negra (IA pura) y caja gris (híbrido).}
	\label{fig:diagrama_caja_negra_blanca}
\end{figure}

\subsection{Optimización: optimalidad vs. factibilidad }

El segundo desafío fundamental radica en el propio algoritmo de búsqueda. La elección de la herramienta de optimización no es arbitraria, sino que está dictada por la naturaleza matemática inherente al problema del despacho de riego. Al formularlo, el problema se revela como un \textbf{Problema de Programación Mixta-Entera No Lineal (MINLP)}, una clasificación con profundas implicaciones computacionales.

 La naturaleza \textbf{Mixta-Entera (MI)} surge de las variables de decisión: el estado de una válvula o bomba en un intervalo de tiempo 
  es una decisión fundamentalmente binaria (0 para cerrado o 1 para abierto). A esto se suma la \textbf{No Linealidad (NL)}, que proviene directamente de la función de evaluación. El modelo hidráulico que predice el caudal y la presión —ya sea de caja blanca o negra— es una función compleja de la física de fluidos, no una simple suma algebraica. 

 La combinación de decisiones binarias (un espacio de búsqueda combinatorio) y una función de evaluación no lineal clasifica este problema en la categoría de complejidad \textbf{NP-hard} \citep{GAschedulingSussex}. Esto significa que es computacionalmente intratable: el tiempo requerido para encontrar la solución óptima garantizada crece exponencialmente a medida que el problema escala (más válvulas o más intervalos). El espacio de soluciones a explorar es demasiado grande.

Esta realidad computacional genera un conflicto directo e irresoluble con las necesidades de un sistema de riego operativo. El estado del arte, si bien ofrece \textit{solvers} exactos para MINLP que buscan el óptimo global, no puede garantizar el tiempo de convergencia. 

Dado este compromiso entre la optimalidad y la factibilidad, la práctica establecida para problemas de \textit{scheduling} (programación) y asignación de recursos NP-hard es el uso de \textbf{metaheurísticas}\citep{GAschedulingSurvey, PerformanceMetaheuristics}. Algoritmos como los Algoritmos Genéticos (GA) proponen un enfoque alternativo: en lugar de buscar una garantía de optimalidad global, ofrecen una \textbf{garantía de tiempo de ejecución}. Están diseñados para encontrar soluciones de alta calidad (casi óptimas) dentro de un presupuesto computacional fijo \citep{ComparativeMetaheuristics}. Para un problema de ingeniería en un entorno productivo, esta capacidad de obtener una solución factible en un tiempo acotado es un requisito operativo fundamental, convirtiendo a esta estrategia en un enfoque viable.
%----------------------------------------------------------------------------------------

\section{Motivación}

La motivación de este trabajo final es proponer una arquitectura de software unificada que resuelva los dos dilemas fundamentales identificados en el estado del arte. Este proyecto se fundamenta en la hipótesis de que es posible lograr tanto la precisión en el modelado como la factibilidad en la optimización, sin sacrificar la robustez física ni el rendimiento operativo.
Ante el desafío del modelado, que obliga a elegir entre la consistencia de la ``caja blanca`` y la precisión de la ``caja negra``, este trabajo implementa un enfoque de ``caja gris`` o Gemelo Digital Híbrido. Esta arquitectura utiliza el simulador físico (basado en EPANET/WNTR) para obtener una línea de base robusta y físicamente coherente. Adicionalmente, emplea un modelo de Inteligencia Artificial (IA) no para modelar el sistema completo, sino con la tarea específica de predecir y corregir el error residual del modelo físico \citep{HybridModelsKaggle, HybridSeriesArxiv}. De esta forma, se busca unificar la consistencia física de los primeros principios con la precisión empírica de los modelos de IA.

El desafío de la optimización (MINLP) también requiere un enfoque específico. Si bien los solvers exactos son la herramienta fundamental para garantizar la optimalidad global, su tiempo de convergencia es, por la naturaleza NP-hard del problema, indefinido. Un cálculo que puede tardar horas o días no es una herramienta útil para la toma de decisiones agronómicas diarias.
Por esta razón, el proyecto encontró un enfoque alternativo que prioriza la velocidad de respuesta. Se seleccionó una metaheurística, el Algoritmo Genético (GA), fundamentada en su probada robustez para problemas de programación complejos \citep{GAschedulingSurvey}. Esta estrategia permite al sistema encontrar soluciones de alta calidad (casi óptimas) dentro de un límite de tiempo fijo, lo cual resulta aceptable y práctico para las necesidades de la operación real.

\section{Objetivos y alcance}

Basado en la problemática y la motivación expuestas, el objetivo general de este trabajo es diseñar, implementar y validar el software que constituye la ``Capa de Gestión y Optimización`` para un sistema de riego solar. El proyecto se enfoca en desarrollar los dos componentes técnicos fundamentales que responden directamente a los dilemas identificados en el estado del arte.

El primer objetivo específico es la creación de un Gemelo Digital Híbrido. Esto implica implementar un simulador físico (basado en EPANET/WNTR) y, de forma crucial, desarrollar un modelo de inteligencia artificial cuya función es predecir y corregir el error residual de dicho modelo físico. Este componente servirá como la función de evaluación precisa y robusta para el optimizador.

El segundo objetivo específico es el diseño de un optimizador metaheurístico. Se implementará un Algoritmo Genético (GA) capaz de navegar el complejo espacio de búsqueda MINLP . Este agente utilizará el Gemelo Digital Híbrido como su función de aptitud para generar cronogramas de despacho casi óptimos, priorizando la factibilidad temporal sobre la optimalidad exacta. El alcance de la implementación incluye también el desarrollo de la función de costos híbrida y unificada, que permite al agrónomo ponderar los objetivos de uniformidad, eficiencia y sostenibilidad del acuífero.

Es importante delimitar que el alcance de este trabajo es exclusivamente de software y se centra en el motor de decisión. El proyecto no incluye el diseño, provisión, instalación o mantenimiento de ningún componente de hardware (como bombas, sensores, PLCs o gateways). Asimismo, el desarrollo de la interfaz de usuario final o ``Capa de Presentación`` (dashboard) y el soporte operativo post-entrega quedan fuera del alcance de esta memoria. Esta arquitectura global, y el foco específico del proyecto, se ilustran en la figura \ref{fig:diagrama_sistema}.

\begin{figure}[ht]
	\centering
	\includegraphics[width=\textwidth]{Figures/despacho_v1.pdf}
	\caption{Diagrama en bloques de la arquitectura completa del sistema de riego.}
	\label{fig:diagrama_sistema}
\end{figure}
\chapter{Introducción específica}
\label{chap:introduccion_especifica}

En este capítulo se presentan los fundamentos teóricos que sustentan la arquitectura de optimización del proyecto. Se detalla el modelado híbrido de ''caja gris'', la formulación matemática del problema como un MINLP, la teoría de los Algoritmos Genéticos como solución metaheurística y el Método de Funciones de Penalización para el manejo de restricciones.

\section{Teoría del modelado híbrido: corrección de residuales}
\label{sec:teoria_modelado_hibrido}

La simulación de sistemas físicos complejos, como lo es un sistema hidráulico, se aborda generalmente desde dos enfoques principales \citep{ref7}.

El primer enfoque es el modelado de ''Caja blanca'' (\textit{White-Box}), de naturaleza determinística, basado en leyes físicas de la mecánica de fluidos. \citep{ref7, ref8}. El motor de simulación hidráulico \texttt{EPANET} es un ejemplo representativo de este enfoque \citep{ref9, ref10}. La ventaja principal de los modelos de caja blanca es su interpretabilidad y la garantía de predicciones físicamente consistentes. Sin embargo, los modelos que se basan únicamente en leyes físicas tienen una limitación fundamental: debido a simplificaciones, inevitablemente producen desviaciones significativas, o sesgos, en comparación con la realidad \citep{ref4, ref5}. Estos sesgos surgen de dinámicas no modeladas y de la incertidumbre en los parámetros intrínsecos, como la rugosidad exacta de las tuberías (el coeficiente $C$ en la ecuación de Hazen-Williams) o las pérdidas de carga menores no contempladas en el modelo .

El segundo enfoque es el modelado de ''Caja negra'' (\textit{Black-Box}). Estos modelos son de naturaleza estocástica, como las redes neuronales (NN) o los perceptrones multicapa (MLP), que mapean entradas a salidas basándose puramente en datos observados \citep{ref7, ref8}. Su fortaleza es la capacidad de modelar relaciones no lineales extremadamente complejas sin necesidad de conocer el modelo físico del sistema \citep{ref6}. Sin embargo, sus desventajas son críticas para sistemas de ingeniería. Primero, requieren volúmenes de datos de entrenamiento que a menudo son muy grandes \citep{ref4, ref5}. Segundo, y más importante, son propensos a generar predicciones que pueden ser físicamente inconsistentes, como violar la conservación de masa, un riesgo inaceptable para un sistema de control de recursos hídricos .

Como solución, en este proyecto se propone un modelado de ''Caja Gris'' (\textit{Gray-Box}), un paradigma que busca combinar las fortalezas de ambos enfoques\citep{ref6}. Específicamente, se implementa la arquitectura de corrección de residuales .

En este paradigma, el modelo de inteligencia artificial no intenta aprender la totalidad del sistema desde cero. En su lugar, el modelo físico ($f_{fisico}$) proporciona la predicción principal y la robustez, mientras que el modelo de IA ($f_{AI}$) tiene una tarea mucho más acotada y factible: aprender y predecir el error residual (el sesgo sistemático) del modelo físico \citep{ref5}.

La formulación matemática de esta arquitectura híbrida es :
\begin{equation}\label{f_hibrido}
f_{hybrid} = f_{fisico} + f_{AI}(\mathbf{X})
\end{equation}
Donde el objetivo de entrenamiento (el \textit{target}) para $f_{AI}$ es el residual $R$, definido como :
\begin{equation}
R = Y_{real} - Y_{fisico}
\end{equation}
La literatura académica lo describe como una herramienta de predicción eficiente \citep{ref4, ref5}, donde el modelo $f_{fisico}$ aporta la consistencia y el $f_{AI}$ aporta la precisión, corrigiendo los sesgos propios del primero .

A continuación se analizan ambas partes de la ecuación \ref{f_hibrido}.

\subsection{Componente físico ($f_{fisico}$)}
\label{subsec:componente_fisico}

El componente físico se desarrolla utilizando el motor de simulación \texttt{EPANET} \citep{ref9, ref10}. Este modelo se rige por leyes físicas claves que definen la complejidad del problema :

\begin{enumerate}
    \item Pérdida de carga por fricción: se utiliza la ecuación de Hazen-Williams, una relación empírica que define la pérdida de carga ($H_f$) en una tubería. 
    \begin{equation}
    H_f = \frac{10.67 \cdot L}{C^{1.852} \cdot D^{4.87}} \cdot Q^{1.852}
    \end{equation}
    
    \item Comportamiento de emisores (goteros): el caudal de salida ($q_e$) en un gotero se modela usando la ley de potencia, que relaciona el caudal con la presión ($p$) en el nodo. En el proyecto se utiliza un exponente $\gamma$ de 0.46, característico del equipamiento instalado, introduciendo otra no linealidad.
    .
    \begin{equation}
    q_e = C_d \cdot p^{\gamma}
    \end{equation}
\end{enumerate}

Además, se emplea la simulación en modo Análisis Dirigido por Presión (PDA). A diferencia del Análisis Dirigido por Demanda (DDA), que asume que la demanda siempre se satisface, el PDA ajusta el caudal entregado si la presión en un nodo cae, ofreciendo una simulación mucho más realista para un sistema de riego solar donde la presión de la bomba varía constantemente.

\subsection{Componente corrector ($f_{AI}$)}
\label{subsec:componente_corrector}

El componente de IA, $f_{AI}$, se implementa como un Perceptrón Multicapa (MLP).

El vector de características $\mathbf{X}$ incluye estados operativos reales (como la presión y nivel freatico del aquifero) y decisiones de control (como el estado de las válvulas). También incluye la propia salida del modelo físico como una característica de entrada. Esto significa que el MLP aprende explícitamente en función de la predicción física, operando como un verdadero corrector de sesgos .

\section{Formulación del problema de optimización}
\label{sec:formulacion_optimizacion}

Una vez que se dispone de una función de aptitud $f_{hybrid}$ de alta fidelidad,es necesario definir formalmente el problema de optimización.

\subsection{Formulación matemática formal}
\label{subsec:formulacion_formal}

El problema de optimización del despacho de riego consiste en encontrar el cronograma de apertura y cierre de válvulas que maximice un conjunto de objetivos, sujeto a las restricciones físicas de la red hidráulica y la disponibilidad de recursos. A continuación, se presenta una nomenclatura y formulación general del modelo .

\begin{table}[h!]
\centering
\caption{Nomenclatura del Modelo de Optimización. }
\label{tab:nomenclatura}
\begin{tabular}{l p{10cm}}
\toprule
\textbf{Símbolo} & \textbf{Definición} \\
\midrule
$\mathbf{V}$ & Matriz de decisiones de despacho (variable de decisión). \\
$v_{i,t}$ & Variable binaria: 1 si la válvula $i$ está abierta en el tiempo $t$, 0 si está cerrada. \\
$\mathcal{T}$ & Horizonte de planificación (ej. 24 horas, discretizado en $T$ intervalos). \\
$\mathcal{I}$ & Conjunto de todas las válvulas (sectores de riego). \\
$f_{hybrid}(\mathbf{V})$ & Función de aptitud (el gemelo digital) que mapea un plan $\mathbf{V}$ a sus consecuencias físicas. \\
$Q_{entregado,i}(\mathbf{V})$ & Caudal total entregado al sector $i$ bajo el plan $\mathbf{V}$. \\
$Q_{objetivo,i}$ & Caudal objetivo para el sector $i$. \\
$P_{consumida,t}(\mathbf{V})$ & Potencia consumida por la bomba en el tiempo $t$ bajo el plan $\mathbf{V}$. \\
$P_{disponible,t}$ & Potencia disponible pronosticada en el tiempo $t$ (ej. solar). \\
$N_{recurso,t}(\mathbf{V})$ & Nivel del recurso (ej. freático) en el tiempo $t$ bajo el plan $\mathbf{V}$. \\
$N_{min}$ & Nivel mínimo de seguridad del recurso. \\
$w_u, w_t, w_a$ & Ponderaciones estratégicas para los objetivos de la función. \\
\bottomrule
\end{tabular}
\end{table}

\subsubsection{Función objetivo ponderada}

Un enfoque común para problemas multiobjetivo es formular una función de aptitud unificada. Esta función es una suma ponderada que puede incluir objetivos agronómicos, de eficiencia y de sostenibilidad :

\begin{equation}
\max_{\mathbf{V}} \left( w_u \cdot \mathcal{U}(\mathbf{V}) - w_t \cdot \mathcal{T}_{bombeo}(\mathbf{V}) - w_a \cdot \mathcal{S}_{acuifero}(\mathbf{V}) \right)
\end{equation}

Donde :
\begin{itemize}
    \item $\mathcal{U}(\mathbf{V})$: Una métrica de uniformidad (ej. minimizar el déficit cuadrático $\sum (Q_{objetivo,i} - Q_{entregado,i})^2$), que se busca maximizar.
    \item $\mathcal{T}_{bombeo}(\mathbf{V})$: Una métrica del tiempo de bombeo (ej. $\sum v_{i,t}$), que se busca minimizar.
    \item $\mathcal{S}_{acuifero}(\mathbf{V})$: Una métrica del estrés sobre el acuífero (ej. la extracción total de agua), que se busca minimizar.
\end{itemize}

\subsubsection{Restricciones del problema}

La maximización está sujeta a un conjunto de restricciones físicas y operativas :

\begin{enumerate}
    \item Restricción de potencia: la potencia consumida por la bomba no puede exceder la potencia disponible en ningún momento.
    \begin{equation}
    P_{consumida,t}(f_{hybrid}(\mathbf{V})) \le P_{disponible,t} \quad \forall t \in \mathcal{T}
    \end{equation}
    
    \item Restricción de sostenibilidad (ej. nivel freático): el nivel del recurso no debe caer por debajo del límite de seguridad.
    \begin{equation}
    N_{recurso,t}(f_{hybrid}(\mathbf{V})) \ge N_{min} \quad \forall t \in \mathcal{T}
    \end{equation}
    
    \item Restricciones hidráulicas (implícitas): las leyes físicas de la red (Hazen-Williams, Ley de Potencia) están encapsuladas dentro de la evaluación de la función $f_{hybrid}$ .
    
    \item Variables de decisión binarias: las decisiones de control de válvulas son binarias.
    \begin{equation}
    v_{i,t} \in \{0, 1\} \quad \forall i \in \mathcal{I}, \forall t \in \mathcal{T}
    \end{equation}
\end{enumerate}



\chapter{Diseño e implementación} % Main chapter title

\label{Chapter3} % Change X to a consecutive number; for referencing this chapter elsewhere, use \ref{ChapterX}

Todos los capítulos deben comenzar con un breve párrafo introductorio que indique cuál es el contenido que se encontrará al leerlo.  La redacción sobre el contenido de la memoria debe hacerse en presente y todo lo referido al proyecto en pasado, siempre de modo impersonal.

\definecolor{mygreen}{rgb}{0,0.6,0}
\definecolor{mygray}{rgb}{0.5,0.5,0.5}
\definecolor{mymauve}{rgb}{0.58,0,0.82}

%%%%%%%%%%%%%%%%%%%%%%%%%%%%%%%%%%%%%%%%%%%%%%%%%%%%%%%%%%%%%%%%%%%%%%%%%%%%%
% parámetros para configurar el formato del código en los entornos lstlisting
%%%%%%%%%%%%%%%%%%%%%%%%%%%%%%%%%%%%%%%%%%%%%%%%%%%%%%%%%%%%%%%%%%%%%%%%%%%%%
\lstset{ %
  backgroundcolor=\color{white},   % choose the background color; you must add \usepackage{color} or \usepackage{xcolor}
  basicstyle=\footnotesize,        % the size of the fonts that are used for the code
  breakatwhitespace=false,         % sets if automatic breaks should only happen at whitespace
  breaklines=true,                 % sets automatic line breaking
  captionpos=b,                    % sets the caption-position to bottom
  commentstyle=\color{mygreen},    % comment style
  deletekeywords={...},            % if you want to delete keywords from the given language
  %escapeinside={\%*}{*)},          % if you want to add LaTeX within your code
  %extendedchars=true,              % lets you use non-ASCII characters; for 8-bits encodings only, does not work with UTF-8
  %frame=single,	                % adds a frame around the code
  keepspaces=true,                 % keeps spaces in text, useful for keeping indentation of code (possibly needs columns=flexible)
  keywordstyle=\color{blue},       % keyword style
  language=[ANSI]C,                % the language of the code
  %otherkeywords={*,...},           % if you want to add more keywords to the set
  numbers=left,                    % where to put the line-numbers; possible values are (none, left, right)
  numbersep=5pt,                   % how far the line-numbers are from the code
  numberstyle=\tiny\color{mygray}, % the style that is used for the line-numbers
  rulecolor=\color{black},         % if not set, the frame-color may be changed on line-breaks within not-black text (e.g. comments (green here))
  showspaces=false,                % show spaces everywhere adding particular underscores; it overrides 'showstringspaces'
  showstringspaces=false,          % underline spaces within strings only
  showtabs=false,                  % show tabs within strings adding particular underscores
  stepnumber=1,                    % the step between two line-numbers. If it's 1, each line will be numbered
  stringstyle=\color{mymauve},     % string literal style
  tabsize=2,	                   % sets default tabsize to 2 spaces
  title=\lstname,                  % show the filename of files included with \lstinputlisting; also try caption instead of title
  morecomment=[s]{/*}{*/}
}



\section{Arquitectura General del Software}
\label{sec:arquitectura_general}

El diseño del sistema de riego se basa en una arquitectura de cuatro capas: la capa física, la de control y adquisición, la de gestión y optimización, y la de presentación (ver figura \ref{fig:diagrama_capas}). Este trabajo se enfoca exclusivamente en la capa de gestión y optimización, que es responsable de transformar los datos históricos y de telemetría en un plan de despacho de riego óptimo.

El software se implementó bajo una arquitectura de microservicios \citep{Newman2015_Microservices}. Esta arquitectura garantiza la separación de responsabilidades y la escalabilidad horizontal de los dos componentes centrales: el entrenamiento continuo del modelo predictivo y su consumo operativo.

El diseño se basa en dos subsistemas, encapsulados en contenedores \textit{Docker} para asegurar un entorno de ejecución aislado y consistente: el subsistema de entrenamiento y ciclo de vida (MLOps) y el subsistema de optimización y servicio. La vinculación entre ellos se establece a través de un registro central de modelos, lo que permite al motor de decisión consumir siempre la versión más precisa del gemelo digital híbrido, mientras el \textit{pipeline} de MLOps opera de forma asíncrona.

\begin{figure}[ht]
    \centering
	\includegraphics[width=\textwidth]{Figures/despacho_v1.pdf}
    \caption{Diagrama de capas del sistema de riego, donde el foco del proyecto es la capa de gestión y optimización.}
    \label{fig:diagrama_capas}
\end{figure}


\subsubsection{Subsistema de entrenamiento y ciclo de vida (MLOps)}
\label{sec:subsistema_mlops}

Este subsistema tiene como objetivo principal la sostenibilidad de la precisión del modelo predictivo. Mantiene el gemelo digital híbrido actualizado al ejecutar un ciclo de vida automatizado que mitiga el (\textit{model drift}) \citep{Schlachter2020_HybridModel}. Este flujo de trabajo, fundamental para un entorno de producción, está orquestado por \textit{pipelines} de \textit{Apache Airflow} \citep{AirflowApache_Orchestration}.

La arquitectura comprende un \textit{stack} de infraestructura para el manejo de los artefactos y la trazabilidad de los experimentos:
\begin{itemize}
    \item Data Lake (\textit{MinIO}): repositorio central de objetos que almacena los datos de telemetría históricos y los \textit{datasets} limpios, actuando como la fuente única de verdad para el entrenamiento.
    \item Registro de Artefactos (\textit{MLflow}): administra el ciclo de vida completo del modelo \citep{MLflowDatabricks}. Registra los hiperparámetros, métricas y los artefactos binarios (el modelo y los transformadores de características) que son promovidos a producción para su consumo por el servicio de inferencia.
    \item Orquestación (\textit{PostgreSQL/Airflow}): la base de datos \textit{PostgreSQL} gestiona los metadatos y el estado de los \textit{pipelines} de \textit{Airflow}, asegurando la fiabilidad y reintentos del proceso de \textit{Integración y Despliegue Continuo} (CI/CD) \citep{Holo2020_MLOps}.
\end{itemize}

El \textit{pipeline} de entrenamiento se estructura en una secuencia de módulos lógicos de procesamiento de datos:

\begin{enumerate}
    \item Módulo de ingesta y consolidación: responsable de la lectura y unificación de los datos de telemetría histórica del campo desde el \textit{Data Lake}.
    \item Módulo de limpieza: filtra los registros inconsistentes y asegura la calidad del \textit{dataset}.
    \item Módulo de ingeniería de características: genera las variables sintéticas necesarias para el modelo de corrección de residuales, como el número de actuadores activos.
    \item Módulo de entrenamiento: ejecuta el entrenamiento. Si el nuevo modelo supera los umbrales de desempeño, es promovido al alias de \textit{"production"} en el registro de modelos, completando el ciclo de integración y despliegue continuo (CI/CD) \citep{Holo2020_MLOps}.
\end{enumerate}

\subsubsection{Subsistema de optimización y servicio}
\label{sec:subsistema_optimizacion}

Este subsistema constituye el corazón operativo del sistema, implementando el motor de decisión. Se diferencian dos partes importantes.
\begin{enumerate}
    \item Capa de servicio de inferencia (\textit{Snapshot API}):
    El diseño de arquitectura resuelve el desafío de rendimiento mediante un microservicio RESTful asíncrono implementado con \textit{FastAPI}. Al iniciar, carga el gemelo digital híbrido, compuesto por el modelo físico \textit{WNTR} \citep{WNTRShaw2018} y el modelo de IA de corrección de residuales, directamente en memoria. Esto transforma el complejo proceso de simulación en una llamada de servicio de baja latencia.
    La \textit{API} está diseñada para ser escalable horizontalmente, utilizando \textit{NGINX} como balanceador de carga. Esto permite al optimizador enviar de manera concurrente miles de peticiones de simulación por segundo, evaluando un gran número de soluciones candidatas en paralelo.

    \item Motor de optimización (\textit{algoritmo genético}):
    El motor implementa el AG. Su conexión con la \textit{Snapshot API} se da a través de una interfaz que distribuye la evaluación de la función de aptitud de cada individuo (plan de riego) a través de las instancias del microservicio de inferencia.
\end{enumerate}


\section{Implementación del gemelo digital híbrido}

\section{Implementación del optimizador de despacho (GA)}

\section{Implementación de funciones de costo y penalizaciones}


% Chapter Template

\chapter{Ensayos y resultados} % Main chapter title

\label{Chapter4} % Change X to a consecutive number; for referencing this chapter elsewhere, use \ref{ChapterX}
Todos los capítulos deben comenzar con un breve párrafo introductorio que indique cuál es el contenido que se encontrará al leerlo.  La redacción sobre el contenido de la memoria debe hacerse en presente y todo lo referido al proyecto en pasado, siempre de modo impersonal.

%----------------------------------------------------------------------------------------
%	SECTION 1
%----------------------------------------------------------------------------------------

\section{Entorno y banco de pruebas}

\section{Pruebas del gemelo digital hibrido}

\section{Caso de estudio: simulación de despacho}



% Chapter Template

\chapter{Conclusiones} % Main chapter title

\label{Chapter5} % Change X to a consecutive number; for referencing this chapter elsewhere, use \ref{ChapterX}
Todos los capítulos deben comenzar con un breve párrafo introductorio que indique cuál es el contenido que se encontrará al leerlo.  La redacción sobre el contenido de la memoria debe hacerse en presente y todo lo referido al proyecto en pasado, siempre de modo impersonal.

\section{Resultados obtenidos}

\section{Trabajo futuro}


%----------------------------------------------------------------------------------------
% Apéndices
%----------------------------------------------------------------------------------------

\appendix

% Incluir apéndices desde archivos separados si es necesario
%\include{Appendices/AppendixA}

%----------------------------------------------------------------------------------------
% Bibliografía
%----------------------------------------------------------------------------------------

\renewcommand{\bibname}{Bibliografía} % Para asegurarte de que el título sea correcto
\phantomsection % Necesario para que el enlace del marcador sea correcto

\printbibliography[heading=bibintoc]

\end{document}






